\documentclass[hyperref={pdfpagelabels=false}]{beamer}
\usepackage{lmodern}
\usepackage{tcolorbox}
\usepackage{algorithm,algorithmic}
\usetheme{Copenhagen}

% the ``Thank you'' page
\def\makethanks{%
  \begin{frame}
  \frametitle{\@thankstitle}
  ?`Preguntas?
  \end{frame}
}
% in ``article'' mode there's no ``Thank you'' page
\mode<article>
  {\providecommand\makethanks{}}

% commands for the title and message of the "Thank you" page
\def\thankstitle#1{
\def\@thankstitle{#1}}
\thankstitle{!`Muchas gracias por su atenci\'on!}

\addtobeamertemplate{navigation symbols}{}{%
    \usebeamerfont{footline}%
    \usebeamercolor[fg]{footline}%
    \hspace{1em}%
    \insertframenumber
}

\title{Desarrollo de una herramienta de visualizaci\'on de redes sociales mediante un enfoque difuso}  
\author{Natalia Mar\'in P\'erez \\
       Tutor: Carlos Gonz\'alez Alvarado} 
\institute{Instituto Tecnol\'ogico de Costa Rica}
\date{Noviembre 27,2017} 
\begin{document}
\begin{frame}
\titlepage
\end{frame}


\begin{frame}
\frametitle{Tabla de contenidos}
\tableofcontents
\end{frame} 


\section{Introducci\'on} 
\begin{frame}
\frametitle{Introducci\'on} 
    \begin{figure}
        \begin{center}
            \includegraphics[height=0.6\textheight]{imagenes/scienceperson.png}
            \hfill
            \includegraphics[height=0.5\textheight]{imagenes/investigacion.png}
        \end{center}
    \end{figure}

\end{frame}
\begin{frame}
\frametitle{Importancia de la difusidad en redes sociales} 
    \begin{figure}
        \begin{center}
            \includegraphics[height=0.35\textheight]{imagenes/clustering.png}
            \hfill
            \includegraphics[height=0.4\textheight]{imagenes/redes.png}

        \end{center}
    \end{figure}

\end{frame}
\subsection{Objetivos}
\begin{frame}
\frametitle{Objetivos}
      \begin{tcolorbox}
  Desarrollar una herramienta de visualizaci\'on de redes sociales utilizando un enfoque difuso. Evaluando los datos de migraci\'on en Costa Rica, as\'i como la representaci\'on de c\'omo interact\'uan las
personas con la tecnolog\'ia.
       \end{tcolorbox}

\begin{itemize}
\item   Analizar y comparar diferentes distancias con el fin de escoger cu\'al se adapta mejor al problema
\item Desarrollar una metodolog\'ia que permita el an\'alisis de cl\'usteres difusos en una red social
\item Implementar una herramienta que represente, de forma difusa, la conexi\'on entre objetos en diferentes particiones
\item Validar la herramienta con diferentes conjuntos de datos para medir su efectividad
\end{itemize} 
\end{frame}


\section{Antecedentes} 
\subsection{Clustering: fuzzy c-means}
\begin{frame}
\frametitle{Algoritmo difuso: fuzzy c-means}
\begin{exampleblock}{F\'ormula fuzzy c-means}
\[
J_m(U,V) = \sum_{k=1}^N \sum_{i=1}^C u_{ij}^m ||Y_k - v_i||^2
\]
\end{exampleblock}
\begin{center}
 \scalebox{0.65}{
\begin{tcolorbox}
\[
Y = {y_1, y_2, ..., y_n} \subset R^n = datos,
\]
\[
c = numero\ de\ clusteres\ en\ Y; 2 \leq c < n,
\]
\[
m = exponente\ en\ peso; 1 \leq m < \infty,
\]
\[
U = particion\ difusa\ de\ Y; U \in M_{fc},
\]
\[
v = v_1, v_2, ..., v_c = vectores\ de\ centros,
\]
\[
v_i = (v_{i1}, v_{i2}, ..., v_{in}) = centro\ de\ cluster\ i
\]
\end{tcolorbox}
}
\end{center}
\end{frame}


\subsection{An\'alisis de redes sociales}
\begin{frame}\frametitle{Concepto de redes sociales y difusidad}
        \begin{figure}[ht]
        \begin{minipage}[b]{0.45\linewidth}
            \centering
            \includegraphics[width=\textwidth]{imagenes/fuerza.png}
            \caption{a. Fuerza de atracci\'on nodo a nodo b. Fuerza de repulsi\'on nodo a nodo}
            \label{fig:a}
        \end{minipage}
        \hspace{0.5cm}
        \begin{minipage}[b]{0.45\linewidth}
            \centering
            \includegraphics[width=\textwidth]{imagenes/cluster.png}
            \caption{Ejemplo de representaci\'on de cluster difuso}
            \label{fig:b}
        \end{minipage}
    \end{figure}
\end{frame}
\begin{frame}\frametitle{Ejemplo de difusidad en redes sociales}
\begin{columns}
    \begin{column}{0.48\textwidth}
            \begin{figure}
    \caption{Movimientos internos en Costa Rica durante una semana-seg\'un encuesta realizada en mayo 2017}
        \begin{center}
            \includegraphics[height=0.6\textheight]{imagenes/prueba.png}
        \end{center}
        \end{figure}
    \end{column}
    \begin{column}{0.48\textwidth}
       \begin{exampleblock}{Movimiento interno}
        \[ t=tiempo\]
        \[ i = provincia_i\]
        \[ j = provincia_j\]
        \[t_i + t_j = 7\]
        
        \[ 
        porcentajes\_membresia=\frac{t_i}{7} y \frac{t_j}{7}
        \]
        \end{exampleblock}
    \end{column}
\end{columns}

    
\end{frame}



\section{Metodolog\'ia}
\subsection{An\'alisis y pre-procesamiento de datos}
\begin{frame}
\frametitle{Proceso general}
    \begin{figure}
        \begin{center}
            \includegraphics[height=0.4\textheight]{imagenes/proceso_general.png}
        \end{center}
    \end{figure}
\end{frame}

\begin{frame}
\frametitle{Caso 1: Datos de migraci\'on en Costa Rica (INEC)}
\begin{columns}
    \begin{column}{0.38\textwidth}
     \begin{figure}
     \includegraphics[height=0.55\textheight]{imagenes/razones.png}
     \end{figure}
    \end{column}
    \begin{column}{0.68\textwidth}
        \begin{table}[]
		\centering
		\caption{Atributos a analizar}
		\label{my-label}
		\begin{tabular}{ll}
		\hline
		\multicolumn{2}{c}{Atributos} \\ \hline
		\begin{tabular}[c]{@{}l@{}}Regi\'on de \\ residencia actual	\end{tabular} & \begin{tabular}[c]{@{}l@{}}Condici\'on de \\ actividad	\end{tabular} \\ \hline
		\begin{tabular}[c]{@{}l@{}}Regi\'on residencia \\ hace dos a\~nos	\end{tabular} & T\'itulo \\ \hline	
		Mantiene familia & Estado conyugal \\ \hline
		\begin{tabular}[c]{@{}l@{}}Ingreso por\\  persona neto\end{tabular} 			& Ocupaci\'on
\end{tabular}
	\end{table}
    \end{column}
\end{columns}

\end{frame}


\begin{frame}
\frametitle{Caso 1: An\'alisis de datos de migraci\'on en Costa Rica}
    \begin{figure}
            \includegraphics[height=0.65\textheight]{imagenes/analisis1.png}
            \hfill
            \includegraphics[height=0.65\textheight]{imagenes/analisis2.png}
    \end{figure}
\end{frame}


\begin{frame}
\frametitle{Caso 2: An\'alisis de datos de interacci\'on de personas con la tecnolog\'ia}
	\begin{columns}
    \begin{column}{0.38\textwidth}
        \begin{figure}
            \includegraphics[height=0.6\textheight]{imagenes/pca.png}
    \end{figure}
    \end{column}
    \begin{column}{0.88\textwidth}
        \begin{table}[]
		\caption{Atributos a analizar}
		\label{my-label}
		\begin{tabular}{ll}
		\hline
		\multicolumn{2}{c}{Atributos} \\ \hline
		\begin{tabular}[c]{@{}l@{}}Uso de internet \\ (general)\end{tabular} 		& \begin{tabular}[c]{@{}l@{}}Tiene tel\'efono \\ inteligente				\end{tabular} \\ \hline
		\begin{tabular}[c]{@{}l@{}}Uso de internet \\ en dispositivo m\'ovil				\end{tabular} & \begin{tabular}[c]{@{}l@{}}Tenencia de \\ computadora\end{tabular} \\ \hline
		\begin{tabular}[c]{@{}l@{}}Uso de internet en \\ el lugar de residencia\end{tabular} & \begin{tabular}[c]{@{}l@{}}Frecuencia en \\ redes sociales\end{tabular} \\ \hline
		\begin{tabular}[c]{@{}l@{}}Nivel de conexi\'on \\ a internet					\end{tabular} & 
		\end{tabular}
		\end{table}
    \end{column}
\end{columns}
    
\end{frame}


\begin{frame}
\frametitle{Pre-procesamiento de datos}
\begin{itemize}
\item Limpieza de datos
\item Transformaci\'on de los datos: normalizaci\'on, discretizaci\'on
\item Obtenci\'on de muestra: solo personas migrantes (interna), mayor de edad.
\end{itemize}
\begin{figure}
        \begin{center}
            \includegraphics[height=0.5\textheight]{imagenes/python_pandas.jpg}
        \end{center}
    \end{figure}  
\end{frame}


\subsection{An\'alisis de distancias}
\begin{frame}
\frametitle{An\'alisis y selecci\'on de distancia}
\begin{itemize}
\item Distancia Ahmad and Dey: comparaci\'on de dos datos categ\'oricos \pause
\item Distancia Euclideana:no aplica para datos categ\'oricos \pause
\item Distancia Hamming: recomendada para an\'alisis de tipo binario \pause
\item Distancia Gower: Permite el an\'alisis de variables mixtas: tanto n\'umericas como categ\'oricas
\end{itemize}

\begin{exampleblock}{F\'ormula Gower}
\[
S_{ij} = \frac{\sum_k w_{ijk} x S_{ijk}}{\sum_k w_{ijk}}
\]
\end{exampleblock}

\end{frame}

\begin{frame}
\frametitle{Distancia de gower}
\begin{center}
 \scalebox{0.65}{
    \begin{minipage}{0.7\linewidth}
\begin{algorithm}[H]
\begin{algorithmic}[1]
\STATE $suma\_elemento\_ij=0$
\STATE $suma\_peso\_ij=0$
\FOR{$c\ en\ columnas$}
\STATE $objeto\_ij=0.0$ 
\STATE $peso\_ij=0.0$ 
        \IF{$tipo\_de\_datos[c]\ es\ numero$}   
        \STATE $objeto\_ij=absoluto(vi[c]-vj[c])/(datos\_mixtos[c])$   
        \STATE $peso\_ij=peso[c]$
        \ELSE   
        \STATE $suma\_elemento\_ij=[1,0][vi[c]==vj[c]]$
        \STATE $peso\_ij=(peso[c])$
        \ENDIF   
        \STATE $suma\_elemento\_ij+= (peso\_ij*objeto_ij)$ 
        \STATE $suma\_peso\_ij+=peso\_ij$    
    \ENDFOR  
    \RETURN $suma\_elemento\_ij/suma\_peso\_ij$
\end{algorithmic}
\caption{Pseudoc\'odigo distancia de gower}
\label{alg:seq}
\end{algorithm}
\end{minipage}%
    }\end{center}
\end{frame}


\subsection{Algoritmo difuso}
\begin{frame}
\frametitle{Proceso algoritmo fuzzy c-means}
    \begin{figure}
        \begin{center}
            \includegraphics[height=0.8\textheight]{imagenes/proceso_difuso.png}
        \end{center}
    \end{figure}
\end{frame}

\begin{frame}
\frametitle{Evaluaci\'on del agrupamiento}
    \begin{figure}
        \begin{center}
            \includegraphics[height=0.4\textheight]{imagenes/silueta1.png}
            \hfill
            \includegraphics[height=0.4\textheight]{imagenes/silueta2.png}
        \end{center}
    \end{figure}
\end{frame}


\subsection{Caracter\'isticas de la visualizaci\'ion}
\begin{frame}
\frametitle{Estructura de archivo para generar visualizacion}
    \begin{figure}
        \begin{center}
            \includegraphics[height=0.7\textheight]{imagenes/estructura_json.png}
            \includegraphics[height=0.6\textheight]{imagenes/estructura_nodos.png}
        \end{center}
    \end{figure}
\end{frame}

\begin{frame}
\frametitle{Comparaci\'on: Enfoque difuso(soft) y k-means (hard) }
    \begin{figure}
    \caption{Comparaci\'on de an\'alisis de agrupaci\'on utilizando algoritmo k-means con respecto al algoritmo de fuzzy c-means}
        \begin{center}
            \includegraphics[height=0.7\textheight]{imagenes/analisismig.png}
        \end{center}
    \end{figure}
\end{frame}


\begin{frame}
\frametitle{Visualizaci\'on de enlace a partir de regiones espec\'ificas }
    \begin{figure}
    \caption{Movimientos y densidad de migraci\'on a partir de la regi\'on Pac\'ifico Central}
        \begin{center}
            \includegraphics[height=0.6\textheight]{imagenes/links.png}
        \end{center}
    \end{figure}
\end{frame}
\section{Validaci\'on} 
\subsection{Visualizacion}
\begin{frame}
\frametitle{Caso 1: Migraci\'on interna en Costa Rica}
\begin{figure}
            \includegraphics[height=0.7\textheight]{imagenes/compara_migracion.png}
    \end{figure}   
\end{frame}


\begin{frame}
\frametitle{Caso 2: Interacci\'on de personas con la tecnolog\'ia en Estados Unidos}
\begin{figure}
        \begin{center}
            \includegraphics[height=0.8\textheight]{imagenes/compara_tec.png}
        \end{center}
    \end{figure}   
\end{frame}



\section{Conclusi\'on}
\begin{frame}
\frametitle{Conclusi\'on}
\begin{itemize}
\item  Evoluci\'on en el tiempo de datos agrupados con un enfoque difuso para una red social
\item Comparaci\'on del an\'alisis difuso con respecto a uno binario
\item Visualizaci\'on propuesta permite una manera intuitiva de inferir la membres\'ia y relaciones de manera din\'amica
\item  El algoritmo \textit{force-directed} para agrupar los nodos de acuerdo a su membres\'ia
\item Casos analizados: Migraciones internas en Costa Rica y encuesta sobre la interacci\'on de personas con la tecnolog\'ia. 
\end{itemize} 
\end{frame}



\makethanks

\section{Material de apoyo} 
\begin{frame}
\frametitle{Material de apoyo}
Diapositivas de apoyo
\end{frame}

\begin{frame}
\frametitle{Trabajo futuro}
\begin{itemize}
\item Probar la metodolog\'ia con otros estudios como: 
\begin{itemize}
\item  el movimiento de las personas a trav\'es del tiempo influido por las nuevas tecnolog\'ias
\item La influencia de las relaciones en la migraci\'on de personas tanto de manera interna como externa
\end{itemize}
\item Se puede mejorar la transici\'on de la visualizaci\'on para que no se note de una manera m\'as natural a trav\'es del tiempo
\end{itemize} 
\end{frame}


\begin{frame}
\frametitle{Similitudes en el conjunto de datos}
\begin{table}[]
\centering
\caption{Par de elementos m\'as similares del conjunto de datos Migraci\'on Interna en Costa Rica}
\label{tabla1}
\begin{tabular}{llllll}
\hline
T\'itulo & \begin{tabular}[c]{@{}l@{}}Condici\'on \\ laboral\end{tabular} & Ingreso & \begin{tabular}[c]{@{}l@{}}Estado \\ conyugal\end{tabular} & \begin{tabular}[c]{@{}l@{}}Calidad \\ vivienda\end{tabular} & \begin{tabular}[c]{@{}l@{}}Mantiene \\ familia\end{tabular} \\ \hline
0.7 & 0 & 1 & 1 & 0 & 1 \\
0.7 & 0 & 1 & 1 & 0 & 0.75 \\ \hline
\end{tabular}
\end{table}
\end{frame}

\begin{frame}
\frametitle{Disimilitudes en el conjunto de datos}

\begin{table}[]
\centering
\caption{Par de elementos m\'as disimilares del conjunto de datos Migraci\'on Interna en Costa Rica}
\label{tabla2}
\begin{tabular}{llllll}
\hline
T\'itulo & \begin{tabular}[c]{@{}l@{}}Condici\'on \\ laboral\end{tabular} & Ingreso  & \begin{tabular}[c]{@{}l@{}}Estado \\ conyugal\end{tabular} & \begin{tabular}[c]{@{}l@{}}Calidad \\ vivienda\end{tabular} & \begin{tabular}[c]{@{}l@{}}Mantiene \\ familia\end{tabular} \\ \hline
1      & 0                   & 1                   & 1               & 0.9                & 1                \\
0      & 0                   & 0.2                  & 6               & 0                & 0             \\ \hline
\end{tabular}
\end{table}

\end{frame}

\begin{frame}
\frametitle{C\'odigo creaci\'on de nodos}
\begin{figure}
        \begin{center}
            \includegraphics[height=0.8\textheight]{imagenes/creacion_nodos.png}
        \end{center}
    \end{figure}   

\end{frame}

\end{document}
